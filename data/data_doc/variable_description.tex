\documentclass{article}
\usepackage[utf8]{inputenc}
\usepackage[colorlinks=True, linkcolor=blue]{hyperref}
\usepackage{tabularx}
\usepackage{booktabs}
\usepackage{longtable}
\usepackage[labelfont=bf]{caption}
\usepackage{mathpazo}

\begin{document}

\begin{longtable}{p{.29\textwidth}  p{.71\textwidth}}
    \caption{Phenotype domains used for clinical metrics}\vspace*{-0,1cm}\\
    \toprule
    \textbf{Phenotype domain} & \textbf{Clinical Variables}\\
    \midrule
\endhead
&\\
Demographics & Age, gender, ethnicity\\
&\\
Admission symptoms & Breathless, chest pain, orthnopea, paroxysmal nocturnal dyspnoea, peripheral oedema, palpitations, syncope\\
&\\
Admission signs & Admission heart rate (HR), admission systolic blood pressure (SBP), admission diastolic blood pressure (DBP), admission mean blood pressure (MAP), admission weight, height, admission body mass index, discharge weight\\
&\\
Risk factors & Atrial fibrillation, hypertension, diabetes, chronic obstructive pulmonary disease, coronary artery disease, history of cerebrovascular disease, hypercholesterolaemia, obstructive sleep apnoea, iron deficiency, obesity\\
&\\
Comorbidities & Depression, dementia, amyloidosis, cancer\\
&\\
12 lead electrocardiogram (ECG) & Rhythm, rate, QRS duration, evidence of atrioventricular (AV) block, T wave inversion (TWI), evidence of left ventricular hypertrophy (LVH), presence of pacemaker\\
&\\
Laboratory tests & Haemoglobin, mean cell volume (MCV), packed cell volume (PCV), white blood cells (WBC), platelets, sodium, potassium, glomerular filtration rate (GFR), albumin, HbA1C, glucose, iron levels, transferrin saturations (TSAT), ferritin, troponin\\
&\\
Echocardiography & Left ventricular ejection fraction (LVEF), left atrial diameter left atrial area, right atrial area, E wave, E deceleration time, Lateral e’, lateral S, E/e’, dilated LV, A wave, E:A, gradient, regional wall motion abnormalities, left ventricular hypertrophy, tricuspid annular planesystolic excursion (TAPSE), pulmonary artery systolic pressure (PASP), mitral regurgitation, tricuspid regurgitation, aortic regurgitation, aortic stenosis\\
&\\
Outcome & Length of stay, time to heart failure hospitalization, time to mortality\\
&\\
\midrule
\end{longtable}

\end{document}