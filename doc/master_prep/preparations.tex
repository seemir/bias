\documentclass{article}
\usepackage[utf8]{inputenc}
\usepackage{mathpazo}
\usepackage{tabularx}
\usepackage[colorlinks=true,citecolor=blue]{hyperref}
\usepackage[round]{natbib}
\usepackage[final]{microtype}

\title{\textbf{Preparations for Master Thesis}}
\author{\large Samir Adrik\\ \normalsize Student number: 975149}
\date{October, 2017 - November, 2017}

\begin{document}

\maketitle

\vspace{-1cm}\section{Practical Preparations}

\noindent \textbf{Language:} The written language for the thesis is \textbf{English}. By doing so one does not need to worry about translation difficulties/challenges. Most of the literature on possible subjects is in also written in English.\\

\noindent \textbf{Version control:} Setup a master repository called \texttt{NMBU\_BIAS\_MasterThesis\_ SamirAdrik} where all the documentation and source code will be stored. The chosen version control system is \textbf{Bitbucket} (can also switch to \textbf{gitHub} if that is preferred), and each of the respected supervisors (Kristin, Håvard and Trygve) shall be given read access to this repository.\\

\noindent \textbf{Programming languages:} The two main programming languages that the source code is to be written in is \textbf{R} (version 3.4.1) and \textbf{Python} (version 3.6.3). The source code will be written using the IDEs \textbf{RStudio} and \textbf{PyCharm}. Furthermore, extensive use of the module \texttt{rpy2} \citep{gautier2008rpy2} will be used to run R embedded in Python processes. Other programming languages such as \textbf{Matlab} are also open for use.\\

\noindent \textbf{Markup and reference management:} The thesis will be written in \textbf{\LaTeX} $\!$ using \textbf{ShareLateX} as IDE and \textbf{bibtex} as reference management software.

\section{Theoretical Preparations}

\noindent Possible methodologies to use in the thesis: The following is a list of the methodologies that have been taught at courses at NMBU and the ones that are currently researched for use in the thesis.\\

\noindent \textbf{Dimension reduction:} Principal Component Analysis (PCA) which was used extensively in STIN300 and STAT340. Also working on getting a better understanding of addition methods for dimension reduction including Kernel PCA, Graph-based kernel PCA, LDA (used in STAT340) and GDA.\\

\noindent \textbf{Cluster Analysis:} Mostly based on methods covered in STAT340. Hierarchical clustering and Kmean clustering are the main methods currently examined.\\

\noindent \textbf{Classification:} Linear Discriminant Analysis (LDA) and Quadratic Discriminant Analysis (QDA) (STAT340). Also working on getting a better understanding of additional methods including, Logistic regression (used in STAT340), Naive Bayes classifier, Support vector machines, k-nearest neighbor (STIN300 and STAT340) and Decision trees (INF221). \\

\noindent \textbf{Multivariate Statistics:}  Multivariate statistical analysis using Tikhonov regularization, PCA/RCR, PLS, Factor analysis and Multivariate analysis of variance (MANOVA). All methods are mentioned in STAT340 with the exception of Tikhonov regularization.

\bibliographystyle{apalike} 
\bibliography{ref}

\end{document}
