\documentclass[../thesis.tex]{subfiles}

\begin{document}

\chapter{Conclusion}
\label{chap:conc}

\noindent In this thesis, we investigated how well various clustering algorithms (hierarchical clustering, k-means and expectation–maximization) perform in producing phenotypically distinct clinical patient groups (i.e. phenomapping) with HFpEF and HFmrEF. Furthermore, we evaluated the performance of various classification algorithms (k-nearest neighbours, logistic regression, naive bayes, linear discriminant analysis, support vector machines and random forest) in predicting the clinical outcomes mortality and readmission of patients with HFpEF and HFmrEF. All the models and techniques were applied on a data set consisting of 375 patients with symptomatic HF identified at a tertiary hospital in the United Kingdom. In the clustering of the patients based on the subtypes HFmrEF and HFpEF, we found that the hierarchical and k-means clustering algorithms show signs of being better at clustering patients with HF compared to the physicians. Overall these algorithms produce 62 significantly different baseline characteristics compared to 59 produced by the physicians. However, if the objective is to use the results to find additional "new clusters", then we cannot say with certainty that the choice of clustering algorithms or the clustering data used (whether it's with or without post-diagnosis) will systematically enhance the "uniqueness" of the patient groups. In the classification of mortality and readmission, we found that the random forest and logistic regression are good candidates. That is, they both rank very high compared to the other algorithms evaluated. The random forest predicted mortality with 72\% accuracy and readmission with 99.7\%. The logistic regression had similar results with approximately 67\% accuracy for mortality and 97.5\% for readmission. Similar results are reported in the literature.


\end{document}