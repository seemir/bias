\documentclass[../thesis.tex]{subfiles}

\begin{document}

\chapter{Conclusion}
\label{chap:conc}

\noindent In this thesis, we attempt to investigate how well various clustering algorithms (hierarchical clustering, k-means and expectation–maximization) perform in producing phenotypically distinct clinical patient groups (i.e. phenomapping) with heart failure with preserved ejection fraction (HFpEF) and mid-range ejection fraction (HFmrEF). Furthermore, we evaluate the performance of various classification algorithms (k-nearest neighbours, logistic regression, naive Bayes, linear discriminant analysis, support vector machines and random forest) in predicting patient mortality and readmission. All the algorithms were applied on a data set consisting of 375 patients with symptomatic heart failure (HF) identified at a tertiary hospital in the United Kingdom.\\
\indent In the clustering of the patients based on the subtypes HFmrEF and HFpEF, we found that the hierarchical and k-means clustering algorithms show signs of being better at clustering patients with HF compared to the physicians. Overall these algorithms produced 62 significantly different baseline characteristics compared to 59 produced by the physicians. However, if the objective is to use the results to find additional "new clusters", then we cannot say with certainty that the choice of clustering algorithms or the clustering data used (whether it's with or without post-diagnosis) will systematically enhance the "uniqueness" of the patient groups.\\
\indent In the classification of mortality and readmission, we found that the random forest and logistic regression show promising potential. That is, the level of accuracy for which the algorithms predicted mortality and readmission rank high compared to the other algorithms evaluated. The random forest predicted mortality with 72\% accuracy and readmission with 99.7\%. The logistic regression had similar results with approximately 67\% accuracy for mortality and 97.5\% for readmission. Similar results are reported in the literature. Our findings lend support to the idea that the application of such algorithms may help in better understanding the complex nature of a clinical syndrome such as HF.


\end{document}