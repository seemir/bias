\documentclass[../thesis.tex]{subfiles}

\begin{document}

\chapter{Introduction}
\label{chap:intro}

\noindent  Heart failure (HF) is a clinical syndrome typically associated with high prevalence, high mortality, frequent hospitalization and overall reduced quality of life (QoL). Approximately 65 million people are effected by HF globally \citep{hay2017global}. With an aging population, it is expected that the prevalence of HF is to increase. In developed countries, about 3-5\% of hospital admissions are linked with HF, accounting for about 2\% of the total health cost \citep{tripoliti2017heart}. It is not unusual for HF to be characterized as a global pandemic with prognosis being worse than that of most cancers, see e.g. \cite{braunwald2015war} and \cite{savarese2017global}.\\
\indent In terms of clinical classification, there is no single "universally agreed upon" system for classifying the causes of HF. Typically HF manifests it self as at least two major subtypes \citep{alonso2015exploring}. All being commonly distinguished based on measures of the left ventricle ejection fraction (LVEF)\footnote{Fraction of blood ejected from the left ventricle of the heart with each contraction. Calculated as the left ventricle stroke volume (LVSV) divided by the left ventricle end-diastolic volume (LVEDV), i.e. $LVEF = LVSV / LVEDS$ \citep{cikes2015beyond}}. The first subtype encompasses patients with LVEF values larger or equal to 50\%. These patients are characterized as patients with HE with preserved ejection fraction (HEpEF). The second subtype includes patients with LVEF values less than 40\%, and are characterized as patients with HE with reduced ejection fraction (HErEF). However, the European Society of Cardiology (ESC) recently defined a third subtype with patients belong to the "gray zone" or the "the middle child", namely when the LVEF values lies between 40\% and 49\%\footnote{The American College of Cardiology Foundation/American Heart Association (ACCF/AHA) were the first to define HF with borderline ejection fraction as being patients with LVEF values between \textbf{41\%} to 49\% \citep{yancy2013}.}. These patients are defined as having HF with mid-range ejection fraction (HFmrEF), see e.g. \cite{lam2014middle} and \cite{ponikowski2016}. Clinically clustering patients according to HF subtypes and identifying HF patients most at risk of mortality and readmission is something that remains challenging. Especially considering that the 1-year mortality rates for acute HF across different regions in Europa ranges from 21.6\% to 36.5\% (35.1\% - 37.5\% in the US), see e.g. \cite{cheng2014outcomes}, \cite{inamdar2016heart} and \cite{crespo2016european}. Patients with HFmrEF have also a clinical profile and prognosis that is close to those of HFpEF (which have LVEF values considered to be normal). Current therapies have also shown to be unable to reduce \textit{both} morbidity and mortality in patients with HFmrEF and HFpEF, see e.g. \cite{ponikowski2016} and \cite{hsu2017heart}. All of which makes the overall job of identifying and distinguishing these patients challenging. It is also unknown if improving phenotypic classification is clinically useful or even possible \citep{shah2014phenomapping}.\\
\indent Nonetheless, the rapid increase in available medical data on patients has lead to machine learning (ML) techniques gaining widespread attention by researchers. The application of such techniques is one that \textit{may} offer an opportunity to build better management strategies, as well as early detection and better prediction of adverse effects associated with HF. Of the ML techniques gaining most attention, one typically finds \textit{clustering} and \textit{classification} methods being intensely studied. Accordingly, the use of ML techniques to identify distinct patient groups with \textit{post-diagnosed} HFmrEF and HFpEF most at risk of mortality and readmission, is one we will try to examine to it full potential.

\section{Problem statement}
\label{sec:prob_stat}

\noindent In this thesis, we investigate how well various clustering methods (hierarchical clustering, k-means and expectation–maximization) perform in producing phenotypically distinct clinical patient groups (i.e. phenomapping) with HFpEF and HFmrEF. Furthermore, we also evaluate the performance of various classification algorithms (k-nearest neighbours, logistic regression, naive bayes, linear discriminant analysis, support vector machines and random forest) in predicting the clinical outcomes (mortality and re-admission) of patients with HFpEF and HFmrEF. When evaluating the results, we compare the clustered according to their level Homogeneity, i.e. the number of statistically significant baseline characteristics within each group and rank methods accordingly. For the classification we evaluate the forecasting estimations based on various statistical properties (accuracy and Cohen's Kappa) and validate with k-fold cross-validation in order to rank methods accordingly. All the models and techniques are applied on a data set consisting of 375 patients with symptomatic HF identified at a tertiary hospital in the United Kingdom.  

\section{Thesis Structure}
\label{sec:thesis_struc}

\noindent The thesis is divided into five chapters and proceeds as follows: The next chapter (\ref{chap:back}) reviews the literature on the application of ML techniques for the assessment of heart failure. This is done to put the proposed research in a relevant context. Chapter (\ref{chap:method}) details the methodology, including presenting the data and the quality of the data. Preliminary analysis on the data will also be done in this chapter. This includes evaluating and selecting a data set based on methods of imputation and dimensional reduction. Next, chapter (\ref{chap:exp}) presents the results of the clustering comparisons with the prediction accuracy of the clinical outcomes, with conclusive remarks and discussion found in chapter (\ref{chap:conc}). The source code and relevant statistical output can be found in the appendix.

\end{document}