
\begin{footnotesize}
\begin{tabularx}{\textwidth}{LLLL}
\caption[HF subtypes based on LVEF]{HF subtypes based on LVEF \citep[page.~2137]{ponikowski2016}}\label{tab:HF_subtypes}\\
\toprule
Criteria & HFrEF & HFmrEF & HFpEF\\
\midrule
\endfirsthead
\caption*{\textbf{Table \ref{tab:HF_subtypes}:} HF subtypes based on LVEF (\textit{continued})}\\
\toprule
Criteria & HFrEF & HFmrEF & HFpEF\\
\midrule
\endhead
1 & Symptoms $\pm$ Signs & Symptoms $\pm$ Signs & Symptoms $\pm$ Signs\\ 
&\\
2 & LVEF $<$ 40\% & 40 $\leq$ LVEF $<$ 50 & 50 $\leq$ LVEF\\
&\\
3 & -- & 1. Elevated NP levels (fig \ref{fig:esc_algo_hf})
& 1. Elevated NP levels (fig \ref{fig:esc_algo_hf}) \\[0,1cm]
& & 2. At least one additional criteria: & 2. At least one additional criteria:\\[0,1cm]
& & \footnotesize a) Relevant structural heart disease\footnotemark &\footnotesize  a) Relevant structural heart disease\\[0,1cm]
& & \footnotesize  b) Diastolic dysfunction\footnotemark & \footnotesize  b) Diastolic dysfunction\\
\midrule
\end{tabularx}
\end{footnotesize}

\footnotetext[7]{\textit{Left ventricular hypertrophy} (LVH): Thickening of the heart muscle of the left ventricle of the heart and/or \textit{Left atrial enlargement} (LAE): Enlargement of the left atrium (LA) of the heart \citep{nagueh2009recommendations}}

\footnotetext{Increased resistance to diastolic filling of one or both cardiac ventricles. In addition to structural abnormalities, physiological derangement of myocardial inactivation and relaxation \citep{grossman1990diastolic}.}