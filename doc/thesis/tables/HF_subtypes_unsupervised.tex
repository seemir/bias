
\begin{footnotesize}
\begin{tabularx}{\textwidth}{LLLL>{\raggedright\arraybackslash}p{3cm}}
\caption{Literature review of HF subtype clustering}\label{tab:ML_HF_subtype_unsupervised_lit}\\
\toprule
Author & Method & Data & Features & Results\\
\midrule
\endfirsthead
\caption*{\textbf{Table \ref{tab:ML_HF_subtype_unsupervised_lit}:} Literature review of HF subtype classification (\textit{continued})}\\
\toprule
Author & Method & Data & Features & Results\\
\midrule
\endhead
\cite{shah2014phenomapping} & Hierarchical, model-based clustering & $N = 397$ with HFpEF & 67 continuous clinical variables & The analysis revealed 3 distinct pheno-groups.\\
&\\
\cite{ahmad2014clinical} & Hierarchical clustering (Ward's minimum variance method) & $N = 2331$ (1619 incl., 712 excl.) & 45 baseline clinical variables &  Four clusters were identified whose patients varied considerably along measures of age, sex, race, symp- toms, comor- bidities, HF etiology, socio- economic status, quality of life, cardiopulmonary exercise testing parameters, and biomarker levels.\\
&\\
\cite{alonso2015exploring} & k-Means clustering, EM, SIBA. & 3 Data sets:\newline D1: $N=48$ (13 HFrEF, 35 HFpEF)\newline\newline D2: $n=63$ (29 HFrEF, 34 HFpEF) \newline\newline D3: $N=403$ (137 HFrEF, 150 HFpEF) & End-systolic Volume Index, End-diastolic volume index & Algorithms generated dividing patterns\\
&\\
\cite{kao2015characterization} & Latent class analysis (LCA) & $N=4113$ with HFpEF & 11 prospect- ively selected clinical features &  Identified 6 subgroups of HFpEF patients with significant differences in event-free survival. \\
&\\
\cite{ahmad2016clinical} & Hierarchical clustering (Ward's minimum variance method) & $N = 433$ (172 incl.) & 29 baseline clinical variables & Four advanced HF clusters were identified. \newline\newline The analysis was done on patients diagnosed with acute decompen- sated heart failure (ADHF).\\
&\\
\cite{katz2017phenomapping} & Hierarchical clustering, model-based clustering & $N=1273$ & 47 continuous clinical variables & Identified 2 distinct groups that differed markedly in clinical characteristics, cardiac structure /function, and indices of cardiac mechanics.\\
\midrule
\end{tabularx}
\end{footnotesize}